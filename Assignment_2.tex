\documentclass[journal,12pt,twocolumn]{IEEEtran}
\usepackage{tikz}
\usepackage{amsmath}
\usepackage{amssymb}
\pagestyle{empty}
\usepackage{setspace}
\singlespacing
\usepackage{caption}
\captionsetup{justification=centering}
\usepackage{amsthm}
\usepackage{amssymb,amsmath}


\begin{document}
\newcommand{\myvec}[1]{\ensuremath{\begin{pmatrix}#1\end{pmatrix}}}
\newcommand{\cmyvec}[1]{\ensuremath{\begin{pmatrix*}[c]#1\end{pmatrix*}}}
\providecommand{\norm}[1]{\lVert#1\rVert}
\newcommand{\mydet}[1]{\ensuremath{\begin{vmatrix}#1\end{vmatrix}}}
\newcommand{\proj}[2]{\textbf{proj}_{\vec{#1}}\vec{#2}}
\newcommand{\abs}[1]{\left\lvert#1\right\rvert}
\newcommand{\RNum}[1]{\uppercase\expandafter{\romannumeral #1\relax}}
\newcommand{\Rnum}[1]{\lowercase\expandafter{\romannumeral #1\relax}}
\let\StandardTheFigure\thefigure
\let\vec\mathbf

\title{
BASICS OF PROGRAMMING

ASSIGNMENT - 2
}
\author{ SHUBHAM RAMESH DANGAT - SM21MTECH14003}
\maketitle
\newpage
\bigskip
\renewcommand{\thefigure}{\theenumi}
\bibliographystyle{IEEEtran}
\section*{ Chapter \RNum{3} Miscellaneous Example-\RNum{6} Q.23}

Show that the area of the triangle formed by the lines whose equations a_s + b_y + c_s = 0,($$ s=1,2,3 $$) is (\Delta)^2/(2C1C2C3)
where \Delta = \mydet{
 a_1 & b_1 & c_1  \\ 
 a_2 & b_2 & c_2  \\
 a_3 & b_3 & c_3 
}
\section*{\textbf{Solution}}
\noindent
Clearly, we can scale the coefficients of a given linear equation by any (non-zero) constant and the result is unchanged. Therefore, by dividing-through by  $$\sqrt{a_i^2+b_i^2}$$ , we may assume our equations are in "normal form":
\begin{align}
x \cos\theta + y \sin\theta - p &= 0 \\
x \cos\phi + y \sin\phi - q &= 0 \\
x \cos\psi + y \sin\psi - r &= 0
\end{align}

with $\theta$, $\phi$, $\psi$ and p, q, r (and A, B, C and a, b, c) as in the figure 2:


Then,
C_1 = \left|\begin{array}{cc}
\cos\phi & \sin\phi \\
\cos\psi & \sin\psi
\end{array} \right| = 

\sin\psi\cos\phi - \cos\psi\sin\phi = \sin(\psi-\phi) = 

\sin \angle ROQ = \sin A

\textsl{}

Likewise,

\textsl{}


C_2 = \sin B \qquad C_3 = \sin C

\textsl{}

Moreover,

\textsl{}

D := \left|\begin{array}{ccc}
\cos\theta & \sin\theta & - p \\
\cos\phi   & \sin\phi   & - q \\
\cos\psi   & \sin\psi   & - r
\end{array}\right| = 

- \left( p C_1 + q C_2 + r C_3) = - \left(\;p \sin A + q \sin B + r \sin C\;\right)

\textsl{}

Writing d for the circumdiameter of the triangle, the Law of Sines tells us that,

\dfrac{a}{\sin A} = \frac{b}{\sin B} = \frac{c}{\sin C} = d

\textsl{}


Therefore,
\begin{align}
D &= - \left( \frac{ap}{d} + \frac{bq}{d} + \frac{cr}{d} \right) \\[4pt]
&= -\frac{1}{d}\left(\;ap + b q + c r\;\right) \\[4pt]
&= -\frac{1}{d}\left(\;2|\triangle COB| + 2|\triangle AOC| + 2|\triangle BOA| \;\right) \\[4pt]
&= -\frac{2\;|\triangle ABC|}{d}
\end{align}

Also,

C_1 C_2 C_3 = \sin A \sin B \sin C = \dfrac{a}{d}\dfrac{b}{d}\sin C


= \dfrac{2\;|\triangle ABC|}{d^2}

\textsl{}

Finally:

\textsl{}

\dfrac{\Delta^2}{2C_1C_2C_3} = \dfrac{4\;|\triangle ABC|^2/d^2}{4\;|\triangle ABC|/d^2} = |\triangle ABC|

\textsl{}

Hence Proved.

\textsl{}

\begin{figure}[ht]
    \centering
\includegraphics[width=\columnwidth]{bop_3.PNG}
\end{figure}
\qquad Figure 1: Actual triangle

\begin{figure}[ht]
    \centering
\includegraphics[width=\columnwidth]{bop_2.PNG}
\qquad Figure 2: Constructions done on above triangle for proof
\end{figure}
\end{document}